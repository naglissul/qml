\documentclass[12pt]{article}
\usepackage{amsmath, amssymb, amsthm}
\usepackage{enumitem}
\usepackage{geometry}
\usepackage{graphicx}
\usepackage[T1]{fontenc}
\usepackage{physics}
\usepackage{tikz}
\usetikzlibrary{quantikz2}

\geometry{a4paper, landscape, twocolumn, margin=0.5cm, top=0.5cm}

\begin{document}

    \textbf{Quantum Computing cheatsheet}
    
    2024-09-10 by Naglis Šuliokas
    
    \begin{enumerate}
        \item Basics: 
        
        $\ket{0} =  \begin{bmatrix} 1\\0 \end{bmatrix}$,
        $\ket{1} = \begin{bmatrix} 0\\1 \end{bmatrix}$,
        $\ket{\pm} = \frac{1}{\sqrt2}(\ket{0} \pm \ket{1})$,
        $iY = ZX$

        \item Phase kickback   
        
        \begin{minipage}{4cm}
            \begin{quantikz}
                \lstick{$\ket{\psi}$} & \ctrl{1} & \qw \rstick{$\ket{\psi_{\theta}}$} \\
                \lstick{$\ket{v}$} & \gate{U} & \qw \rstick{$\ket{v}$} \\
            \end{quantikz}
        \end{minipage}
        \begin{minipage}{5cm}
            $U\ket{v} = e^{i\theta} \ket{v}$
            
            $\psi$ is in superposition
        \end{minipage}
        
        \item Superdense coding

        $\ket{\Phi^+} = \text{CNOT} \, ( H \otimes I ) \ket{0} \otimes \ket{0} = \frac{1}{\sqrt{2}} ( \ket{00} + \ket{11} )$, encodes 00

        $\ket{\Phi^-} = \hspace{2cm} (Z \otimes I) \ket{\Phi^+} = \frac{1}{\sqrt{2}} ( \ket{00} - \ket{11} )$, encodes 01

        $\ket{\Psi^+} = \hspace{1.9cm} (X \otimes I) \ket{\Phi^+} = \frac{1}{\sqrt{2}} ( \ket{01} + \ket{10} )$, encodes 10

        $\ket{\Psi^-} = \hspace{1.8cm} (iY \otimes I)\ket{\Phi^+} = \frac{1}{\sqrt{2}} ( \ket{01} - \ket{10} )$, encodes 11
        
        \item Oracles
        
            \begin{minipage}{5cm}
                \begin{quantikz}
                    \lstick{$\ket{x}$} & \ctrl{1} & \qw \rstick{$\ket{x}$} \\
                    \lstick{$\ket{y}$} & \gate{U_f} & \qw \rstick{$\ket{y \oplus f(x)}$}
                \end{quantikz}
            \end{minipage}
            \begin{minipage}{5cm}
                \begin{quantikz}
                    \lstick{$\ket{x}$} & \ctrl{1} & \qw \rstick{$\ket{x}$} \\
                    \lstick{$\ket{0}$} & \gate{U_f} & \qw \rstick{$\ket{f(x)}$}
                \end{quantikz}
            \end{minipage}
            \begin{minipage}{5cm}
                \begin{quantikz}
                    \lstick{$\ket{x}$} & \ctrl{1} & \qw \rstick{$(-1)^{f(x)}\ket{x}$} \\
                    \lstick{$\ket{-}$} & \gate{U_f} & \qw \rstick{$\ket{-}$}
                \end{quantikz}
            \end{minipage}
        
        \item Deutsch's algorithm
        \item Deutsch-Jozsa algorithm
        
        $\displaystyle H^{\otimes n} \ket{0}^{\otimes n} = \frac{1}{\sqrt{2^n}} \sum_{x \in \{0, 1\}^n} \ket{x}$
        
        $\displaystyle H^{\otimes n} \ket{x} = \frac{1}{\sqrt{2^n}} \sum_{z \in \{0, 1\}^n} (-1)^{x \cdot z}\ket{z}$
    \end{enumerate}

\tiny
References: 
\begin{enumerate}
    \item Quantum Soar on YouTube.
    \item Nielsen, M. A., \& Chuang, I. L. (2001). Quantum computation and quantum information (Vol. 2). Cambridge: Cambridge university press.
\end{enumerate}

\end{document}

